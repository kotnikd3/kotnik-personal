%%%%%%%%%%%%%%%%%%%%%%%%%%%%%%%%%%%%%%%%%
% "ModernCV" CV and Cover Letter
% LaTeX Template
% Version 1.11 (19/6/14)
%
% This template has been downloaded from:
% http://www.LaTeXTemplates.com
%
% Original author:
% Xavier Danaux (xdanaux@gmail.com)
%
% License:
% CC BY-NC-SA 3.0 (http://creativecommons.org/licenses/by-nc-sa/3.0/)
%
% Important note:
% This template requires the moderncv.cls and .sty files to be in the same 
% directory as this .tex file. These files provide the resume style and themes 
% used for structuring the document.
%
%%%%%%%%%%%%%%%%%%%%%%%%%%%%%%%%%%%%%%%%%

%----------------------------------------------------------------------------------------
%	PACKAGES AND OTHER DOCUMENT CONFIGURATIONS
%----------------------------------------------------------------------------------------

\documentclass[11pt,a4paper,roman]{moderncv} % Font sizes: 10, 11, or 12; paper sizes: a4paper, letterpaper, a5paper, legalpaper, executivepaper or landscape; font families: sans or roman

\usepackage[utf8x]{inputenc}
\usepackage[slovene]{babel}
\selectlanguage{slovene}

\moderncvstyle{classic} % CV theme - options include: 'casual' (default), 'classic', 'oldstyle' and 'banking'
\moderncvcolor{blue} % CV color - options include: 'blue' (default), 'orange', 'green', 'red', 'purple', 'grey' and 'black'

\usepackage{lipsum} % Used for inserting dummy 'Lorem ipsum' text into the template

\usepackage[scale=0.75]{geometry} % Reduce document margins
\setlength{\hintscolumnwidth}{2.5cm} % Uncomment to change the width of the dates column
\setlength{\makecvtitlenamewidth}{6.5cm} % For the 'classic' style, uncomment to adjust the width of the space allocated to your name
%----------------------------------------------------------------------------------------
%	NAME AND CONTACT INFORMATION SECTION
%----------------------------------------------------------------------------------------

\firstname{Denis} % Your first name
\familyname{Kotnik} % Your last name

% All information in this block is optional, comment out any lines you don't need
\title{Curriculum Vitae}
\datum{$\ast$ 13. april 1991, Novo mesto}
\address{Jezero 3A, 8210 Trebnje}{Slovenija}
\mobile{041-628-416}
% \phone{(000) 111 1112}
% \fax{(000) 111 1113}
\email{denis.kotnik@gmail.com}
\homepage{www.kotnikd.com}{www.kotnikd.com} % The first argument is the url for the clickable link, the second argument is the url displayed in the template - this allows special characters to be displayed such as the tilde in this example
% \extrainfo{additional information}
\photo[110pt][0.4pt]{pictures/denis} % The first bracket is the picture height, the second is the thickness of the frame around the picture (0pt for no frame)
\quote{"Najdaljša je pot, ki te pripelje najbližje k sebi, in najtežja je vaja, ki rodi najpreprostejši napev." \textemdash Rabindranath Tagore}

%----------------------------------------------------------------------------------------

\begin{document}

\makecvtitle % Print the CV title
%\pagestyle{empty} % Odstrani nogo.

%----------------------------------------------------------------------------------------
%	EDUCATION SECTION
%----------------------------------------------------------------------------------------

%\section{O meni}
%\cvitem{Ime in priimek}{Denis Kotnik}
%\cvitem{Datum rojstva}{13. april 1991}
%\cvitem{Kraj rojstva}{Novo mesto}
%\cvitem{Prebivališče}{Jezero 3A, 8210 Trebnje}
%\cvitem{Telefon}{041-628-416}
%\cvitem{Email}{denis.kotnik@gmail.com}
%\cvitem{Spletna stran}{www.kotnikd.com}

\section{Formalna izobrazba}
%% VEJICE NI
%\xpatchcmd\cventry{,}{}{}{}
\renewcommand*{\cventry}[7][.25em]{%
  \cvitem[#1]{#2}{%
    {\bfseries#3}%
    %\ifthenelse{\equal{#4}{}}{}{, {\slshape#4}}% I changed this line (with comma) ...
    \ifthenelse{\equal{#4}{}}{}{ {\slshape#4}}% ... into this one (without comma).
    \ifthenelse{\equal{#5}{}}{}{, #5}%
    \ifthenelse{\equal{#6}{}}{}{, #6}%
    .\strut%
    \ifx&#7&%
      \else{\newline{}\begin{minipage}[t]{\linewidth}\small#7\end{minipage}}\fi}}
%%%

\cventry{2014\textendash danes}{}{\textit{Magistrski študijski program druge stopnje Računalništvo in informatika}}{Univerza v Ljubljani}{Fakulteta za računalništvo in informatiko}{}  % Arguments not required can be left empty

%% VEJICA PONOVNO JE
%\xpatchcmd\cventry{,}{}{}{}
\renewcommand*{\cventry}[7][.25em]{%
  \cvitem[#1]{#2}{%
    {\bfseries#3}%
    \ifthenelse{\equal{#4}{}}{}{, {\slshape#4}}% I changed this line (with comma) ...
    %\ifthenelse{\equal{#4}{}}{}{ {\slshape#4}}% ... into this one (without comma).
    \ifthenelse{\equal{#5}{}}{}{, #5}%
    \ifthenelse{\equal{#6}{}}{}{, #6}%
    .\strut%
    \ifx&#7&%
      \else{\newline{}\begin{minipage}[t]{\linewidth}\small#7\end{minipage}}\fi}}
%%%


\cventry{2010\textendash 2014}{Diplomirani inženir računalništva in informatike (VS)}{\textit{Visokošolski strokovni študijski program Računalništvo in informatika}}{Univerza v Ljubljani}{Fakulteta za računalnistvo in informatiko}{}
\cventry{2006\textendash 2010}{Elektrotehnik računalništva}{\textit{Srednje-poklicno izobraževanje}}{Šolski center Novo mesto}{Srednja elektro šola in tehniška gimnazija}{}


\section{Magistrsko delo (v nastajanju)}
\cvitem{Naslov}{\emph{Podatkovni tokovi in rezervoarsko vzorčenje pri napovedovanju proizvodnje sončnih elektrarn}}
\cvitem{Mentor}{izr. prof. dr. Matjaž Kukar}
\cvitem{Opis}{Cilj magistrskega dela je napoved proizvodnje električne energije sončnih elektrarn z uporabo algoritmov za delo s podatkovnimi tokovi.}

\section{Diplomsko delo}

\cvitem{Naslov}{\emph{Prilagodljivo kratkoročno napovedovanje lokalnih vremenskih parametrov}}
\cvitem{Mentor}{izr. prof. dr. Matjaž Kukar}
\cvitem{Opis}{Cilj diplomskega dela je preizkusiti, ali lahko z uporabo regresijskih metod ali umetnih nevronskih mrež popravimo oziroma izboljšamo napoved hitrosti vetra za določeno območje Slovenije.}
\cvitem{Povezava}{\url{http://eprints.fri.uni-lj.si/2721/}}

\clearpage % Nova stran.

%----------------------------------------------------------------------------------------
%	WORK EXPERIENCE SECTION
%----------------------------------------------------------------------------------------

\section{Izkušnje}
\cvitem{}{}
\subsection{\textbf{CGS plus d.o.o}., \textit{Brnčičeva ulica 13, 1000 Ljubljana}}
\cvitem{Trajanje}{17. februar 2014\textendash danes, 1213 ur (do konca 2016)}
\cvitem{Opis}{
Na oddelku \textit{Okolje} sem se do sedaj ukvarjal predvsem z infrastrukturo cestnovremenskih postaj \textit{Družbe za avtoceste v Republiki Sloveniji (DARS)}, \textit{Direkcije Republike Slovenije za infrastrukturo (DRSI)} in nekaterih slovenskih občin. Postaje sem vzdrževal, kakšno pa tudi sestavil in sprogramiral (na 1 tedenskem izobraževanju v nemškem podjetju \textit{OTT Hydromet} sem pridobil certifikat programiranja njihovih podatkovnikov).
\newline
\newline
V programskem jeziku \textit{Python} in \textit{C} sem za nekatere cestnovremenske postaje razvijal vmesnik za mini računalnik \textit{Nanos G20}, ki skrbi za komunikacijo med \textit{RS232}, \textit{RS485} in \textit{ethernet} merilniki.
\newline
Največ strokovnega znanja pa sem pridobil na področju razvoja in vzdrževanja Microsoftovih tehnologij, kot so \textit{ASP.NET 4} spletne aplikacije za namene vizualizacij podatkov (\textit{MVC, C\#, Javascript, HTML, XML, CSS}), \textit{SQL Server 2014/2016}, Windows sistemske storitve in \textit{IIS}.
\newline
Sodeloval sem tudi na projektu \textit{METRoSTAT E!7050}, katerega cilj je bil popraviti napovedi temperatur cestišč z uporabo umetnih nevronskih mrež v programskem jeziku \textit{R}.
}
\cvitem{}{}

\subsection{\textbf{Famnit, Univerza na Primorskem}, \textit{Glagoljaška 8, 6000 Koper}}
\cvitem{Trajanje}{1. april\textendash 31. julij 2015, 160 ur}
\cvitem{Opis}{
V sodelovanju s \textit{Fakulteto za matematiko, naravoslovje in informacijske tehnologije} in podjetjem \textit{CGS plus} sem bil del projekta \textit{Po kreativni poti do praktičnega znanja}.
\newline
V projektu sem razvil in skrbel sem za spletno aplikacijo, v kateri je bil poudarek na vizualizaciji podatkov v obliki grafov. Aplikacijo sem razvil v \textit{ASP.NET 4} (\textit{MVC}, \textit{C\#}).
}
\cvitem{}{}

\subsection{\textbf{Comtrade d.o.o.}, \textit{Letališka cesta 29B, 1000 Ljubljana}}
\cvitem{Trajanje}{8.\textendash 19. julij 2013, 80 ur}
\cvitem{Opis}{
Cilj poletne šole \textit{Bodi smart, bodi eko} je bil razviti sistem za krmiljenje naprav v pametnih hišah.
\newline
V ekipi sem prevzel vlogo \textit{backend} razvijalca na računalniku \textit{Raspberry Pi} oz. vzpostavil in vzdrževal podatkovno bazo ter razvil vmesnik v programskem jeziku \textit{PHP} za komunikacijo z \textit{Android} napravami.
}

\clearpage % Nova stran.

\section{Izobraževanja, tečaji in ostalo}
\cvitem{2016\textendash danes}{Pomočnik administratorja omrežja, \textit{Rožna dolina, blok 12}}
\cvitem{2015}{Introduction to game development, \textit{www.coursera.org}}
\cvitem{2014}{Machine learning, \textit{www.coursera.org}}
\cvitem{2012\textendash 2014}{Član mednarodne organizacije EESTEC, \textit{www.eestec-lj.org}}
\cvitem{2013}{Learning how to learn, \textit{www.coursera.org}}
\cvitem{2013}{Izpit za voditelja čolna do dolžine 24m}
\cvitem{2014}{Introduction to music production, \textit{www.coursera.org}}
\cvitem{2013}{An introduction to interactive programming in Python, \textit{www.coursera.org}}
\cvitem{2010}{CISCO CCNA Exploration: Network Fundamentals}
\cvitem{Otroška leta}{Dokončana osnovna glasbena šola Trebnje}

%----------------------------------------------------------------------------------------
%	LANGUAGES SECTION
%----------------------------------------------------------------------------------------

\section{Jeziki}

\cvitem{Slovenščina}{Materni jezik}
\cvitem{Angleščina}{Odlično razumevanje, prav dobro govorjenje, dobro pisanje}
\cvitem{Hrvaščina}{Dobro razumevanje, dobro govorjenje, dobro pisanje}


\section{Interesi}

\renewcommand{\listitemsymbol}{\textendash~} % Changes the symbol used for lists

\cvlistdoubleitem{Filozofija}{Psihoanaliza}
\cvlistdoubleitem{Znanost}{Računalništvo}
\cvlistdoubleitem{Igranje kitare}{Solo petje in petje v zboru}
\cvlistdoubleitem{Tek}{Borilna veščina Wing Chun}
\cvlistdoubleitem{Hribolazništvo}{Odbojka}
\cvlistdoubleitem{Smučanje}{Badminton}
\cvlistitem{Šah}

\section{Osebne lastnosti}
\cvlistdoubleitem{Odprtomiseln}{Introvertiran}
\cvlistdoubleitem{Strah pred avtoritetami}{Čut za skrb, odgovornost}
\cvlistdoubleitem{Všeč mi je red in čistoča}{Včasih len}
\cvlistdoubleitem{Organiziran}{Všeč so mi osebni izzivi}
\cvlistdoubleitem{Ni mi všeč večinsko, težim k unikatnemu mišljenju}{Se ne znam ceniti - vedno se hočem izpopolniti}

%----------------------------------------------------------------------------------------
%	COVER LETTER
%----------------------------------------------------------------------------------------

% To remove the cover letter, comment out this entire block

%\clearpage

%\recipient{HR Department}{Corporation\\123 Pleasant Lane\\12345 City, State} % Letter recipient
%\date{\today} % Letter date
%\opening{Dear Sir or Madam,} % Opening greeting
%\closing{Sincerely yours,} % Closing phrase
%\enclosure[Priloga]{curriculum vit\ae{}} % List of enclosed documents

%\makelettertitle % Print letter title
%%\pagestyle{empty} % Odstrani nogo.

%\lipsum[1-3] % Dummy text

%\makeletterclosing % Print letter signature

\nopagenumbers % Odstrani stevilke strani oz. ne uposteva te strani.
%----------------------------------------------------------------------------------------

\end{document}