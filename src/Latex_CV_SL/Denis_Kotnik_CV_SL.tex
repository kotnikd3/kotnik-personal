%%%%%%%%%%%%%%%%%%%%%%%%%%%%%%%%%%%%%%%%%
% "ModernCV" CV and Cover Letter
% LaTeX Template
% Version 1.11 (19/6/14)
%
% This template has been downloaded from:
% http://www.LaTeXTemplates.com
%
% Original author:
% Xavier Danaux (xdanaux@gmail.com)
%
% License:
% CC BY-NC-SA 3.0 (http://creativecommons.org/licenses/by-nc-sa/3.0/)
%
% Important note:
% This template requires the moderncv.cls and .sty files to be in the same 
% directory as this .tex file. These files provide the resume style and themes 
% used for structuring the document.
%
%%%%%%%%%%%%%%%%%%%%%%%%%%%%%%%%%%%%%%%%%

%----------------------------------------------------------------------------------------
%	PACKAGES AND OTHER DOCUMENT CONFIGURATIONS
%----------------------------------------------------------------------------------------

\documentclass[11pt,a4paper,roman]{moderncv} % Font sizes: 10, 11, or 12; paper sizes: a4paper, letterpaper, a5paper, legalpaper, executivepaper or landscape; font families: sans or roman

\usepackage[utf8x]{inputenc}
\usepackage[slovene]{babel}
\selectlanguage{slovene}

\moderncvstyle{classic} % CV theme - options include: 'casual' (default), 'classic', 'oldstyle' and 'banking'
\moderncvcolor{blue} % CV color - options include: 'blue' (default), 'orange', 'green', 'red', 'purple', 'grey' and 'black'

\usepackage{lipsum} % Used for inserting dummy 'Lorem ipsum' text into the template

\usepackage[scale=0.75]{geometry} % Reduce document margins
\setlength{\hintscolumnwidth}{2.5cm} % Uncomment to change the width of the dates column
\setlength{\makecvtitlenamewidth}{6.5cm} % For the 'classic' style, uncomment to adjust the width of the space allocated to your name
%----------------------------------------------------------------------------------------
%	NAME AND CONTACT INFORMATION SECTION
%----------------------------------------------------------------------------------------

\firstname{Denis} % Your first name
\familyname{Kotnik} % Your last name

% All information in this block is optional, comment out any lines you don't need
\title{Curriculum Vit\ae\ \normalsize{(\LaTeX)}}
\datum{$\ast$ 13. april 1991, Novo mesto}
\address{Slovenija}{}
\mobile{041-628-416}
% \phone{(000) 111 1112}
% \fax{(000) 111 1113}
\email{denis.kotnik@gmail.com}
\homepage{www.kotnik.si}{www.kotnik.si} % The first argument is the url for the clickable link, the second argument is the url displayed in the template - this allows special characters to be displayed such as the tilde in this example
% \extrainfo{additional information}
\photo[110pt][0.4pt]{pictures/Denis_2020} % The first bracket is the picture height, the second is the thickness of the frame around the picture (0pt for no frame)
%\quote{,,Najdaljša je pot, ki te pripelje najbližje k sebi,
%\newline
%in najtežja je vaja, ki rodi najpreprostejši napev.`` \textemdash Rabindranath Tagore}

%----------------------------------------------------------------------------------------

\begin{document}

\makecvtitle % Print the CV title
%\pagestyle{empty} % Odstrani nogo.

%----------------------------------------------------------------------------------------
%	EDUCATION SECTION
%----------------------------------------------------------------------------------------

%\section{O meni}
%\cvitem{Ime in priimek}{Denis Kotnik}
%\cvitem{Datum rojstva}{13. april 1991}
%\cvitem{Kraj rojstva}{Novo mesto}
%\cvitem{Prebivališče}{Jezero 3A, 8210 Trebnje}
%\cvitem{Telefon}{041-628-416}
%\cvitem{Email}{denis.kotnik@gmail.com}
%\cvitem{Spletna stran}{www.kotnikd.com}

\section{Formalna izobrazba}
%% VEJICE NI
%\xpatchcmd\cventry{,}{}{}{}
\renewcommand*{\cventry}[7][.25em]{%
  \cvitem[#1]{#2}{%
    {\bfseries#3}%
    %\ifthenelse{\equal{#4}{}}{}{, {\slshape#4}}% I changed this line (with comma) ...
    \ifthenelse{\equal{#4}{}}{}{ {\slshape#4}}% ... into this one (without comma).
    \ifthenelse{\equal{#5}{}}{}{, #5}%
    \ifthenelse{\equal{#6}{}}{}{, #6}%
    .\strut%
    \ifx&#7&%
      \else{\newline{}\begin{minipage}[t]{\linewidth}\small#7\end{minipage}}\fi}}
%%%

%% VEJICA PONOVNO JE
%\xpatchcmd\cventry{,}{}{}{}
\renewcommand*{\cventry}[7][.25em]{%
  \cvitem[#1]{#2}{%
    {\bfseries#3}%
    \ifthenelse{\equal{#4}{}}{}{, {\slshape#4}}% I changed this line (with comma) ...
    %\ifthenelse{\equal{#4}{}}{}{ {\slshape#4}}% ... into this one (without comma).
    \ifthenelse{\equal{#5}{}}{}{, #5}%
    \ifthenelse{\equal{#6}{}}{}{, #6}%
    .\strut%
    \ifx&#7&%
      \else{\newline{}\begin{minipage}[t]{\linewidth}\small#7\end{minipage}}\fi}}
%%%

\cventry{2014\textendash 2018}{Magister inženir računalništva in informatike}{\textit{Magistrski študijski program druge stopnje Računalništvo in informatika (smer umetna inteligenca, strojno učenje)}}{Univerza v Ljubljani}{Fakulteta za računalništvo in informatiko}{}  % Arguments not required can be left empty
\cventry{2010\textendash 2014}{Diplomirani inženir računalništva in informatike (VS)}{\textit{Visokošolski strokovni študijski program Računalništvo in informatika (smer informacijski sistemi)}}{Univerza v Ljubljani}{Fakulteta za računalnistvo in informatiko}{}
\cventry{2006\textendash 2010}{Elektrotehnik računalništva}{\textit{Srednje-poklicno izobraževanje}}{Šolski center Novo mesto}{Srednja elektro šola in tehniška gimnazija}{}


\section{Magistrsko delo}
\cvitem{Naslov}{\emph{Podatkovni tokovi in rezervoarsko vzorčenje pri napovedovanju proizvodnje sončnih elektrarn}}
\cvitem{Mentor}{izr.\ prof.\ dr.\ Matjaž Kukar}
\cvitem{Opis}{Cilj magistrskega dela je napoved proizvodnje električne energije sončnih elektrarn z uporabo inkrementalnih modelov strojnega učenja, ki se učijo na spreminjajočih se podatkovnih tokovih.}
\cvitem{Povezava}{\url{https://repozitorij.uni-lj.si/IzpisGradiva.php?id=100897}}

\section{Diplomsko delo}

\cvitem{Naslov}{\emph{Prilagodljivo kratkoročno napovedovanje lokalnih vremenskih parametrov}}
\cvitem{Mentor}{izr.\ prof.\ dr.\ Matjaž Kukar}
\cvitem{Opis}{Cilj diplomskega dela je napoved hitrosti vetra za območje Slovenije z uporabo klasičnih modelov strojnega učenja.}
\cvitem{Povezava}{\url{https://repozitorij.uni-lj.si/IzpisGradiva.php?id=68903}}

\clearpage % Nova stran.

%----------------------------------------------------------------------------------------
%	WORK EXPERIENCE SECTION
%----------------------------------------------------------------------------------------
\section{Tehnologije, ki me zanimajo ...}
\cvitem{}{
Sem uporabnik operacijskega sistema \textit{Linux}. Razvijam v objektno orientiranih programskih jezikih, predvsem \textit{Java}, \textit{Python}, \textit{C\#} in \textit{PHP}. Težim k čisti in berljivi kodi ter principom \textit{SOLID} (\textit{Robert C. Martin: Clean Code}, \textit{Clean Architecture}, diagrami \textit{UML}).
\newline
\newline
Imam izkušnje iz knjižnic in ogrodij \textit{Microsoft .NET} (programi, sistemske storitve Windows), \textit{Scikit-learn} (\textit{Python}) ter \textit{Laravel} (\textit{PHP}, \textit{Vue.js} ter \textit{Bulma}). Pri razvoju sem uporabljal predvsem relacijske (\textit{MySQL} in \textit{Microsoft SQL}), nekaj malega pa tudi nerelacijske podatkovne baze (\textit{Apache Cassandra}).
\newline
\newline
V prihodnosti bi se po možnosti rad osredotočil in podučil iz področja mikrostoritvene ali dogodkovno usmerjene javanske arhitekture (\textit{Apache Kafka}, \textit{Akka}, \textit{Dropwizard}, \textit{Spring}, \textit{Maven}, itd.), tehnologij velikih podatkov (\textit{Apache Spark}, itd.) ali podatkovnega rudarjenja (\textit{Scikit-learn}, \textit{Scikit-multiflow}, \textit{TensorFlow}, \textit{Bayesova statistika}, itd.).
}
\vspace*{0.5cm}

\section{CGS Labs d.o.o., \textit{Brnčičeva ulica 13, 1000 Ljubljana}}
\cvitem{Obdobje}{2014\textendash 2018\textendash danes (študentsko delo, zaposlitev za nedoločen čas)}
\vspace*{0.25cm}
\cvitem{}{
Na oddelku Okolje, na podjetju CGS Labs, sem vodja razvoja informacijskih in odločitvenih sistemov za podporo v cestni meteorologiji oz. pri zimskemu vzdrževanju. V članku na naslednji povezavi smo s sodelavci napisali povzetek omenjenih sistemov.
\newline
\url{https://zenodo.org/record/1485684/files/Contribution_11014_fullpaper.pdf}
\newline
\newline
Razvijam programsko opremo ter vodim ekipo veliko do 4 razvijalcev. Bil sem programski arhitekt naslednje generacije sistema z imenom \textit{Vedra}, pri čemer sem se osredotočil na mikrostoritveno arhitekturo z nerelacijskim podatkovnim modelom (trenutno še v razvoju).
\newline
\newline
V okviru evropskega projekta \textit{METRoSTAT} (\textit{E! 7050}, program \textit{Eurostars}) sem razvil regresijski model strojnega učenja za napovedovanje temperature cestišča. Prispeval sem tudi idejo napovedovanja linijske temperature cestišča z uporabo inkrementalnih algoritmov strojnega učenja na podatkovnih tokovih, ki se sedaj realizira v okviru evropskega projekta \textit{i-RBF} (\textit{Inteligent Route Based Forecast}, program \textit{Eureka}).
\newline
\newline
Moje delo je tudi tesno sodelovanje z uporabniki razvitih rešitev, med drugimi tudi z \textit{Družbo za avtoceste v Republiki Sloveniji} (\textit{DARS}), \textit{Direkcijo Republike Slovenije za infrastrukturo} (\textit{DRSI}) in \textit{Agencijo Republike Slovenije za okolje} (\textit{ARSO}).
}


\clearpage % Nova stran.

\section{Ostali projekti}
\subsection{\textbf{Famnit, Univerza na Primorskem}, \textit{Glagoljaška 8, 6000 Koper}}
\cvitem{Obdobje}{april\textendash avgust 2015}
\cvitem{}{
V sodelovanju s \textit{Fakulteto za matematiko, naravoslovje in informacijske tehnologije} in podjetjem \textit{CGS Labs} sem bil del projekta \textit{Po kreativni poti do praktičnega znanja}.
\newline
V projektu sem razvil spletno aplikacijo v \textit{ASP.NET Webforms}, v kateri je bil poudarek na vizualizaciji podatkov.
}

\vspace*{0.5cm}

\subsection{\textbf{Comtrade d.o.o.}, \textit{Letališka cesta 29B, 1000 Ljubljana}}
\cvitem{Obdobje}{julij 2013}
\cvitem{}{
Z ekipo smo razvili sistem za krmiljenje naprav v pametnih hišah.
\newline
Prevzel sem vlogo razvijalca v zaledju in sicer na računalniku \textit{Raspberry Pi}. Vzpostavil sem podatkovno bazo ter v programskem jeziku \textit{PHP} razvil spletno storitev za komunikacijo z napravami \textit{Android}.
}

\vspace*{0.5cm}

\section{Certifikati}
\cvitem{2017}{Microsoft Certified Professional, \textit{Microsoft Exam 483: Programming in C\#}}
\cvitem{2010}{CISCO CCNA Exploration: Network Fundamentals}

\vspace*{0.5cm}

\section{Tečaji \textit{Coursera}}
\cvitem{2020}{Introduction to Git and Github, \textit{Google}}
\cvitem{2014, 2020}{Machine learning, \textit{Stanford University}}
\cvitem{2020}{Sequences, Time Series and Prediction, \textit{deaplearning.ai}}
\cvitem{2020}{Structuring Machine Learning Projects, \textit{deaplearning.ai}}
\cvitem{2019}{Service oriented architecture, \textit{University of Alberta}}
\cvitem{2018}{Microservices - Fundamentals, \textit{IBM}}
\cvitem{2017}{Front-End Web UI Frameworkds and Tools, \textit{The Hong Kong University}}
\cvitem{2015}{Introduction to game development, \textit{Michigan State University}, projekta:}
\cvitem{}{\listitemsymbol WebGL, igra: \url{https://www.kotnik.si/broadside_pirates_webgl/}}
\cvitem{}{\listitemsymbol Unity3D, video: \url{https://www.youtube.com/watch?v=UZjLVe7J1jM&t=84s}}
\cvitem{2013}{Learning how to learn, \textit{McMaster University}}
\cvitem{2014}{Introduction to music production, \textit{Berklee College of Music}}
\cvitem{2013}{An introduction to interactive programming in Python (part 1), \textit{Rice University}}


\clearpage % Nova stran.

\section{Ostalo}
\cvitem{2010\textendash danes}{Uporabnik operacijskega sistema Linux Ubuntu}
\cvitem{2020}{Alpinistična šola Železničar, \textit{www.aozeleznicar.org}}
\cvitem{2016\textendash 2017}{Pomočnik administratorja omrežja, \textit{Rožna dolina, blok 12}}
\cvitem{2012\textendash 2014}{Član mednarodne organizacije EESTEC, \textit{www.eestec-lj.org}}
\cvitem{2013}{Izpit za voditelja čolna do dolžine 24 m}
\cvitem{Otroška leta}{Osnovna glasbena šola Trebnje}

\vspace*{0.5cm}
%----------------------------------------------------------------------------------------
%	LANGUAGES SECTION
%----------------------------------------------------------------------------------------

\section{Jeziki}

\cvitem{Slovenščina}{Materni jezik}
\cvitem{Angleščina}{Odlično razumevanje, prav dobro govorjenje, prav dobro pisanje, nivo \textit{B2.2}}
\cvitem{Nizozemščina}{600 besed in osnove gramatike, nivo \textit{A1.2}}

\vspace*{0.5cm}

\section{Interesi}

\renewcommand{\listitemsymbol}{\textendash~} % Changes the symbol used for lists

\cvlistdoubleitem{\textbf{Filozofija}}{\textbf{Psihoanaliza}}
\cvlistdoubleitem{Alpinizem, športno plezanje}{Borilne veščine}
\cvlistdoubleitem{Igranje kitare}{Petje (v zboru)}
\cvlistdoubleitem{Odbojka}{Smučanje}
\cvlistdoubleitem{Badminton}{Šah}

\vspace*{0.5cm}

\section{Osebne lastnosti}
\cvlistdoubleitem{Odprtomiseln}{Introvertiran}
\cvlistdoubleitem{Včasih len}{Čut za skrb}
\cvlistdoubleitem{Odgovoren} {Organiziran}
\cvlistdoubleitem{Všeč sta mi red in čistoča}{Všeč so mi osebni izzivi}

%----------------------------------------------------------------------------------------
%	COVER LETTER
%----------------------------------------------------------------------------------------

% To remove the cover letter, comment out this entire block

%\clearpage

%\recipient{HR Department}{Corporation\\123 Pleasant Lane\\12345 City, State} % Letter recipient
%\date{\today} % Letter date
%\opening{Dear Sir or Madam,} % Opening greeting
%\closing{Sincerely yours,} % Closing phrase
%\enclosure[Priloga]{curriculum vit\ae{}} % List of enclosed documents

%\makelettertitle % Print letter title
%%\pagestyle{empty} % Odstrani nogo.

%\lipsum[1-3] % Dummy text

%\makeletterclosing % Print letter signature

\nopagenumbers % Odstrani stevilke strani oz. ne uposteva te strani.
%----------------------------------------------------------------------------------------

\end{document}